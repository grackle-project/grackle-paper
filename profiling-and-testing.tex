\section{Profiling and Testing} \label{sec:profiling-and-testing}



\subsection{Testing Framework}

Testing a library like Grackle, with its mix of Fortran, C, and C++ code, can be
difficult. In the interest of ultimately improving test coverage and make it
easier to prototype tests, Grackle employs a python-based testing framework that
makes heavy use of the \texttt{pygrackle} python wrapper for the Grackle.

Currently there are two different types of tests in the Grackle test suite: unit
tests and answer tests. A unit test in Grackle compares the results of a
calculation using Grackle to some set of ``correct'' answers that are known
\textot{a priori}. The unit tests currently implemented in grackle test that the
unit system is behaving correctly (see Section~\ref{sec:unit-test}) and that the
ionization equilibrium for a primordial gas agrees with the analytic prediction
using the rates implemented in Grackle (see Section~\ref{sec:primordial-test}).

The answer tests consist of a set of example calculations where each calculation
writes out a summary plot as well as an \texttt{hdf5} dataset that is loadable
by the \texttt{yt} package. Known ``correct'' answers for the summary plot and
\textit{yt}-loadable dataset are saved in the repository so that code changes in
Grackle can be tested to ensure that andwers produced by the library do not
change. This process does not prevent *incorrect* answers from being generated
initially, but it does prevent the answers from changing as the code is
developed. Incorrect answers are prevented by manually inspecting test answers
when the test is first added to the codebase. If subsequently a bug is
discovered and the test output changes, then the test answer must also be
manually updated. Currently Grackle contains answer tests calculations for the
cooling rate (see Section~\ref{sec:cooling-rate-test}) in a uniform density
cloud at a fixed time, a calculation of the density and temperature of a gas
cloud undergoing free-fall collapse (see Section~\ref{free-fall-test}), and a
calculation for the temperature evolution of a uniform-density cloud (see
Section~\ref{sec:uniform-cooling-test}.  The answers test are run several times
using different input physics to ensure Grackle's solvers are well-exercised by
the tests.

\subsubsection{unit test: units}
\label{sec:unit-test}

\subsubsection{unit test: ionization equailibrium of a primordial gas}
\label{sec:primordial-test}

\subsubsection{Answer test: cooling rate test}
\label{sec:cooling-rate-test}

\subsubsection{Answer test: free-fall cooling test}
\label{sec:free-fall-test}

\subsubsection{Answer test: uniform cooling test}
\label{sec:uniform-cooling-test}

\subsubsection{Other tests?}


\subsection{Performance}


\subsubsection{Optimization Strategy}


\subsubsection{Speed Test}
