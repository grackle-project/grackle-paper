\section{Summary} \label{sec:summary}

\subsection{Applicability and Limitations}

It is important to remember that the range of physical conditions in
which Grackle can be considered valid is not unlimited.  In Figure
\ref{fig:valid_range}, we show the rough density range over which
different components of the solver is valid.  The high density limit
of the non-equilibrium solver is set roughly by the reactions present
in the network and the range over which their rate coefficients are
trusted.  The limits on the cooling tables correspond to the density
range over which they were calculated with.  Users are especially
cautioned against exceeding the upper density limit of the tabulated
cooling solver.  The critical density above which energy levels are
populated according to LTE exceeds the upper limit of the tables for
many metal coolants \citep{2008MNRAS.385.1443S}.  Thus, these tables
do not capture the NLTE to LTE transition where the cooling rates
change from scaling as $\rho^{2}$ to $\rho$.  Hence, extrapolation
beyond the upper limit will result in vast overprediction of the
cooling rate.  If the use-case requires exceeding this limit, then the
high density metal table should be used in conjunction with the
non-equilibrium solver.  It is also extremely important to remember
that all Grackle calculations are based on the assumption that the
medium is optically thin.  In practice, the length scale of optical
thickness will become very short as density increases.

\begin{figure}
  \centering
  \includegraphics[width=0.48\textwidth]{valid_range.pdf}
  \caption{
    The appropriate density range for different versions of the
    Grackle solver.  The high density metal cooling table (bottom)
    explicitly spans the density range, 10$^{-6}$ cm $^{-3}$ $< n <$
    10$^{12}$ cm $^{-3}$, but extrapolation down to $n = 10^{-10}$ cm
    $^{-3}$ is still valid, as indicated by the dashed line.  For each
    of these, the valid temperature range is roughly 1 K to 10$^{9}$
    K.
  } \label{fig:valid-range}
\end{figure}

The addition of radiative cooling to a simulation creates another
relevant length scale which must be kept in mind.  The cooling length,
defined as the product of the local sound speed and the cooling time,
sets the approximate size of objects as they cool and condense
\citep{2009A&A...508..725I}.  The cooling length is inversely
proportional to density, effectively setting a density limit for
any given spatial resolution.  When this scale becomes unresolved,
radiative losses will be overpredicted, leading to unphysically high
densities and further exacerbating the resolution problem in a runaway
cycle.  This effect is likely related to the over-cooling problem that
has classically plagued cosmological simulations
\citep[e.g.][]{1996ApJS..105...19K, 2001MNRAS.326.1228B}.
In Figure \ref{fig:cooling-length}, we show an estimate of the
cooling length for the scenario of a gas at solar metallicity in a
\citet{2012ApJ...746..125H} radiation background at z = 0, noting how
quickly the cooling length drops below 1 kpc, and even 1 pc, for
densities relevant to galaxy formation simulations.  This
length scale should be taken into consideration when choosing the
density threshold above which sub-grid models are applied.

\begin{figure*}
  \centering
  \includegraphics[width=0.84\textwidth]{cooling_length.pdf}
  \caption{
    The cooling length, defined as the product of the Jeans length and
    the cooling time, as a function of number density and temperature
    for a gas with solar metallicity exposed to a radiation field
    defined by the model of \citet{2012ApJ...746..125H} at z = 0.  The
    narrow line extending from the middle, left to the bottom, right
    represents the temperature where heating and cooling are
    balanced.  Above this line, the gas is being cooled while below
    the line it is being heated.
  } \label{fig:cooling-length}
\end{figure*}

\subsection{Simulation codes with Grackle}
To date, the following codes are known to have \texttt{Grackle}
implemented:
\begin{itemize}

\item ART-I \citep{1999PhDT........25K, 2002ApJ...571..563K}

\item ART-II \citep{2008ApJ...672...19R}

\item CHANGA \citep{2004NewA....9..137W, 2006MNRAS.373.1074S}

\item Enzo \citep{2014ApJS..211...19B}

\item Gadget-3 \citep{2005MNRAS.364.1105S}

\item GAMER \citep{2010ApJS..186..457S}

\item GASOLINE \citep{2004NewA....9..137W}

\item Gear \citep{2012A&A...538A..82R, 2012ASPC..453..141R}

\item Gizmo \citep{2015MNRAS.450...53H}

\item RAMSES \citep{2002A&A...385..337T}

\end{itemize}

\subsection{Future Directions} \label{Future_Directions}

\subsubsection{Including new rates and models in Grackle}
The current code structure is highly integrated. This makes introducing new rates for the 
chemical network or cooling network a rather intricate task requiring multiple changes throughout the code. 
Apart from the fact that this is more time consuming it is also much more error prone. In a future release of the 
code the modularity of the code will be increased greatly. There will be a function to populate the species 
rate coefficients and a function to populate the cooling coefficients. Separate template files can then be 
updated by a developer wishing to use their own rates. This file can then be included in the build and a flag
set to indicate the new rates be used in place of the old rates. Furthermore, a similar method will be 
implemented for solving the network. A template network solver will be available which the developer can use to 
implement a new network with a developer determined number of species. The developer will be responsible for
updating only three files to achieve a solution to their own chemical network. 

\subsubsection{Connection to Radiative Transfer}

As discussed in \S\ref{section:radback}, Grackle includes the effects
of a radiation background that can ionize, heat, and dissociate atoms
and molecules through the radiative reaction and destruction rates in
Equations (\ref{eq:rate_general}) and (\ref{eq:rateCD}), respectively.
Because photo-effects are already implemented into the network, it will
be relatively trivial to couple Grackle with a radiation transport
solver.  Here, Grackle would accept additional spatially-varying
fields of photoionization rates, photoheating rates, and
photodissociation rates to add to or replace the background radiation
values.  All of the heavy computation calculating these radiation
fields is accomplished by the transport solver, whether it be through
a Monte Carlo or moment method.  Grackle can be called in radiative
transfer codes that are either called in post-processing, usually for
cosmic reionization calculations \citep[e.g.][]{2014MNRAS.439..725I,
  2016MNRAS.459.2342M}, or coupled with the hydrodynamics, which is
becoming more commonplace in galaxy and star formation simulations
\citep[e.g.][]{2014MNRAS.442.2560W, 2015MNRAS.451...34R,
  2015arXiv151100011O, 2016arXiv160300034P, 2016arXiv160703117R}.
Additional connections to radiative transfer models could use the
local radiation intensity instead of spatially invariant values.  Some
examples include a more accurate computation of (i) the photoelectric
effect, (ii) the heating and cooling rates from metals, where the
local UV/X-ray flux would be an additional interpolation variable in
the lookup table, and (iii) the self-shielding of a UV background.

\subsubsection{Multiple element cooling}

\subsubsection{Other heating sources?}

%%% Local Variables:
%%% mode: latex
%%% TeX-master: "ms"
%%% End:
