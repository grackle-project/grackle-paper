\section{Introduction} \label{sec:intro}

\subsection{Why is chemistry/radiative cooling necessary in
  simulations?}

\bwo{I will fix the refererences later, ADS is painfully slow right now.}

Modeling of chemistry and radiative cooling is absolutely essential in
a wide range of astrophysical phenomena.  At a fundamental level,
virtually all astrophysical objects begin as a cloud of diffuse plasma
in a gravitational potential that is created primarily either by the
plasma itself (i.e., stars) or by a dark matter halo (i.e.,
cosmologial structure).  In these situations, in the absense of any
additional physical processes the plasma will arrange itself so that
pressure is in rough equilibrium with gravity, and no further
evolution will occur without some external influence.  Some physical
process that allows this plasma to lose energy is necessary in order
to break this stalemate, thus allowing the formation of stars and
galaxies.  The process that typically enables this energy loss is
radiative cooling, often facilitated by a series of chemical reactions
that further
enhance the plasma's ability to lose energy.

Radiative cooling plays a critical role in several important
astrophysical processes.  There is a complex interplay of gas- and
dust-phase chemistry in star formation, which completely dominates the
dynamics of the evolving pre-stellar cloud, and may strongly influence
the resulting stellar initial mass function.  \addcite{ABN02; Turk
2011; Glover+} At a smaller physical scale, radiative cooling can
profoundly affect the structure and behavior of accretion disks around
stars and compact objects.  \addcite{Shakura+Sunyaev 1973}.  The shape
of the `cooling curve' of diffuse astrophysical plasmas -- i.e., the
cooling rate as a function of density and temperature -- is
responsible for the thermal instability that generates a multiphase
interstellar medium \addcite{McKee and Ostriker 1974; Spitzer ISM
book}.  In cosmological structure formation, gas collapsing into dark
matter halos generally experiences a strong shock at roughly the
virial radius, which heats the gas to roughly the virial temperature.
Optically-thin radiative cooling of this plasma allows gas to
concentrate at the center of dark matter halos, and ultimately to form
molecular clouds and stars.  \addcite{Rees+Ostriker 1977; White and
Frenk 1991}

Chemistry is often an important component of the evolution of
astrophysical plasmas.  The creation of simple molecules via gas- and
dust-phase chemical reactions can greatly enhance the efficacy of
cooling. \addcite{e.g., Hollenbach \& McKee 1979; Omukai+ 2005} The
formation and destruction of simple molecules can be energetically
important in some circumstances, such as Population III star
formation.  \addcite{Glover+Abel} And, from a dynamical perspective,
the non-equilibrium evolution of particular ions can have a strong
effect on the ability of gas to cool \addcite{Abel, Anninos, etc.} and
on common observational quantities such as the column density of OVI
absorption line systems in the intergalactic medium that are used to
estimate the metal content of the intergalactic medium and to trace
the ``missing baryons.''  \addcite{Smith+2011; Shull+ something; some
paper on non-eq OVI} 

\subsection{Why is a multi-code library a good thing?}
