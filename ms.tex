%%%%%%%%%%%%%%%%%%%%%%%%%%%%%%%%%%%%%%%%%%%%%%%%%%
% Basic setup. Most papers should leave these options alone.
\documentclass[a4paper,fleqn,usenatbib]{mnras}

% MNRAS is set in Times font. If you don't have this installed (most LaTeX
% installations will be fine) or prefer the old Computer Modern fonts, comment
% out the following line
%\usepackage{newtxtext,newtxmath}
% Depending on your LaTeX fonts installation, you might get better results with one of these:
%\usepackage{mathptmx}
\usepackage{amsmath}	% Advanced maths commands
\usepackage{txfonts}

\setcounter{secnumdepth}{4}

% Use vector fonts, so it zooms properly in on-screen viewing software
% Don't change these lines unless you know what you are doing
\usepackage[T1]{fontenc}
\usepackage{ae,aecompl}


%%%%% AUTHORS - PLACE YOUR OWN PACKAGES HERE %%%%%

% Only include extra packages if you really need them. Common packages are:
\usepackage{graphicx}	% Including figure files
\usepackage{amssymb}	% Extra maths symbols

%%%%%%%%%%%%%%%%%%%%%%%%%%%%%%%%%%%%%%%%%%%%%%%%%%

%%%%% AUTHORS - PLACE YOUR OWN COMMANDS HERE %%%%%

% Please keep new commands to a minimum, and use \newcommand not \def to avoid
% overwriting existing commands. Example:
%\newcommand{\pcm}{\,cm$^{-2}$}	% per cm-squared
\newcommand{\jr}[1]{{\color{red} \bf JR: #1}}   %John Regan
\newcommand{\bwo}[1]{{\color{red} \bf BWO: #1}}   %Brian O'Shea
\newcommand{\addcite}[1]{{\color{blue} (\bf REF: #1)}}   %

%%%%%%%%%%%%%%%%%%%%%%%%%%%%%%%%%%%%%%%%%%%%%%%%%%

%%%%%%%%%%%%%%%%%%% TITLE PAGE %%%%%%%%%%%%%%%%%%%

% Title of the paper, and the short title which is used in the headers.
% Keep the title short and informative.
\title[The Grackle]{Introducing Grackle: a Chemistry and Cooling
  Library for Everyone}

% The list of authors, and the short list which is used in the headers.
% If you need two or more lines of authors, add an extra line using \newauthor
\author[B.D. Smith et al.]
       {Britton D. Smith,$^{1}$\thanks{E-mail: brittonsmith@gmail.com
           (BDS)},
        Greg L. Bryan$^{2,3}$,
        Simon C. O. Glover$^{4}$,
        Nathan J. Goldbaum$^{5}$, \newauthor
        Matthew J. Turk$^{5}$,
        John Regan$^{6,7}$,
        John H. Wise$^{8}$,
        Hsi-Yu Schive$^{5}$,
        Tom Abel$^{11,12}$, \newauthor
        Andrew Emerick$^{2,9}$,
        Brian W. O'Shea$^{14,15,16}$,
        Peter Anninos$^{10}$, \newauthor
        Cameron B. Hummels$^{13}$,
        and Sadegh Khochfar$^{1}$\\
% List of institutions
$^{1}$Institute for Astronomy, University of Edinburgh, Royal
Observatory, Edinburgh EH9 3HJ, UK\\
$^{2}$Columbia University, Department of Astronomy, New York, NY,
10025, USA\\
$^{3}$Simons Center for Computational Astrophysics, New York, NY,
USA\\
$^{4}$Universit\"{a}t Heidelberg, Zentrum f\"{u}r Astronomie, Institut
f\"{u}r Theoretische Astrophysik, Albert-Ueberle-Stra{\ss}e 2, \\69120
Heidelberg, Germany\\
$^{5}$National Center for Supercomputing Applications, University of
Illinois, Urbana-Champaign, IL, 61820, USA\\
$^{6}$Institute for Computational Cosmology, Durham University, South
Road, Durham, DH1 3LE, UK \\
$^{7}$School of Mathematical Sciences, Dublin City University, Dublin,
Ireland \\
$^{8}$Center for Relativistic Astrophysics, Georgia Institute of
Technology, 837 State Street, Atlanta, GA 30332, USA\\
$^{9}$American Museum of Natural History, Department of Astrophysics,
New York, NY, USA\\
$^{10}$Lawrence Livermore National Laboratory, Livermore, CA 94550\\
$^{11}$Kavli Institute for Particle Astrophysics and Cosmology,
Stanford University, Menlo Park, CA 94025, USA\\
$^{12}$Department of Physics, Stanford University, Stanford, CA 94305,
USA\\
$^{13}$California Institute of Technology, Pasadena, CA 91125, USA\\
$^{14}$Department of Physics and Astronomy, Michigan State University,
East Lansing, MI 48824, USA\\
$^{15}$Department of Computational Mathematics, Science and
Engineering, East Lansing, MI 48824, USA\\
$^{16}$JINA: Joint Institute for Nuclear Astrophysics, USA\\
}

% These dates will be filled out by the publisher
\date{Accepted XXX. Received YYY; in original form ZZZ}

% Enter the current year, for the copyright statements etc.
\pubyear{2016}

% Don't change these lines
\begin{document}
\label{firstpage}
\pagerange{\pageref{firstpage}--\pageref{lastpage}}
\maketitle

% Abstract of the paper
\begin{abstract}
We present the \texttt{Grackle} chemistry and cooling library for
astrophysical simulations and models.  \texttt{Grackle} provides
a treatment of non-equilibrium primordial chemistry and cooling for H, D, and He
species, including H$_{2}$ formation on dust grains; tabulated
primordial and metal cooling; multiple UV background models; and
support for radiation transfer and arbitrary heat sources.  The
library has an easily implementable interface for simulation codes
written in C, C++, and Fortran as well as a Python interface with
added convenience functions for semi-analytical models.  As an
open-source project, \texttt{Grackle} provides a community resource
for accessing and disseminating astrochemical data and numerical
methods.  We present the full details of the core functionality, the
simulation and Python interfaces, testing infrastructure, performance,
and range of applicability.
\end{abstract}

% Select between one and six entries from the list of approved keywords.
% Don't make up new ones.
\begin{keywords}
astrochemistry -- methods: numerical -- galaxies: formation
\end{keywords}

%%%%%%%%%%%%%%%%%%%%%%%%%%%%%%%%%%%%%%%%%%%%%%%%%%

%%%%%%%%%%%%%%%%% BODY OF PAPER %%%%%%%%%%%%%%%%%%

\section{Introduction} \label{sec:intro}

\subsection{Why is chemistry/radiative cooling necessary in
  simulations?}

Modeling of plasma chemistry and radiative cooling is absolutely
essential in a wide range of astrophysical phenomena.  At a
fundamental level, virtually all astrophysical objects begin as a
cloud of diffuse plasma in a gravitational potential that is created
primarily either by the plasma itself (i.e., stars) or by a dark
matter halo (i.e., cosmological structure).  In these situations, in
the absense of any additional physical processes the plasma will
arrange itself so that pressure is in rough equilibrium with gravity,
and no further evolution will occur without some external influence.
Some physical process that allows this plasma to lose energy is
necessary in order to break this stalemate, thus allowing the
formation of stars and galaxies.  The process that typically enables
this energy loss is radiative cooling, often facilitated by a series
of chemical reactions that further enhance the plasma's ability to
lose energy.

Radiative cooling plays a critical role in several important
astrophysical processes.  There is a complex interplay of gas- and
dust-phase chemistry in star formation, which completely dominates the
dynamics of the evolving pre-stellar cloud, and may strongly influence
the resulting stellar initial mass function.
\citep{2002Sci...295...93A,2011ApJ...726...55T,2008AIPC..990...25G} At
a smaller physical scale, radiative cooling can profoundly affect the
structure and behavior of accretion disks around stars and compact
objects.  \citep{1973A&A....24..337S} The shape of the `cooling curve'
of diffuse astrophysical plasmas -- i.e., the cooling rate as a
function of density and temperature -- is responsible for the thermal
instability that generates a multiphase interstellar medium
\citep{1977ApJ...218..148M,1978ppim.book.....S}.  In cosmological
structure formation, gas collapsing into dark matter halos generally
experiences a strong shock at roughly the virial radius, which heats
the gas to roughly the virial temperature.  Optically-thin radiative
cooling of this plasma allows gas to concentrate at the center of dark
matter halos, and ultimately to form molecular clouds and stars.
\citep{1977MNRAS.179..541R,1991ApJ...379...52W}

Chemistry is often an important component of the evolution of
astrophysical plasmas.  The creation of simple molecules via gas- and
dust-phase chemical reactions can greatly enhance the efficacy of
cooling. \citep{1979ApJS...41..555H,2005ApJ...626..627O} The formation
and destruction of simple molecules can be energetically important in
some circumstances, such as Population III star formation.
\citep{1998ApJ...508..141O,2002Sci...295...93A,2008MNRAS.388.1627G,2009Sci...325..601T}
And, from a dynamical perspective, the non-equilibrium evolution of
particular ions can have a strong effect on the ability of gas to cool
\citep{1997NewA....2..181A,1997NewA....2..209A} and on common
observational quantities such as the column density of OVI absorption
line systems in the intergalactic medium that are used to estimate the
metal content of the intergalactic medium and to trace the ``missing
baryons.'' \citep{2006ApJ...650..573C,2011ApJ...731....6S,2013MNRAS.434.1043O,2014ApJ...796...49S}


\subsection{Why is a multi-code library a good thing?}


\section{Chemistry and Cooling Methods} \label{sec:physics}



\subsection{Primordial Chemistry}
The treatment of primordial chemistry (i.e.\ the chemistry of metal-free gas) used in Grackle is closely based on the
treatment in the Enzo AMR code \citep{2014ApJS..211...19B}, although Grackle accounts for a few processes that are not included 
in the latest version of Enzo available at the time of writing (version 2.4). The Enzo primordial chemistry itself is based 
originally on the work of \citet{1997NewA....2..181A} and \citet{1997NewA....2..209A}, although the current version has been
modified substantially compared to the original Abel~et~al.\ network. In this section, we describe in detail the chemistry 
included in Grackle and discuss how the resulting set of chemical rate equations is solved. 

\subsubsection{Chemistry Network}
Grackle provides three different primordial chemistry networks, differing in the number of chemical species that they include. 
The six-species network is designed for modelling atomic and/or ionized gas, and tracks the abundances of the species
H, H$^{+}$, He, He$^{+}$, He$^{++}$ and e$^{-}$. The reactions included in this network are listed in Table ***, along with
the sources used for the various reaction rate coefficients. 

\begin{table}
\begin{tabular}{lclcc}
\hline
\multicolumn{3}{c}{Reaction} & \multicolumn{2}{c}{Reference} \\
& & & Data & Fit \\
\hline
${\rm H + e^{-}}$ & $\rightarrow$ & $\rm{H^{+} + e^{-} + e^{-}}$ & 1 & 2 \\
${\rm H^{+} + e^{-}}$ & $\rightarrow$ & $\rm{H + \gamma} $ & 3 & 2, 4\\
${\rm He + e^{-}}$ & $\rightarrow$ & ${\rm He^{+} + e^{-} + e^{-}}$ & 1 & 2 \\
${\rm He^{+} + e^{-}}$ & $\rightarrow$ & ${\rm He + \gamma}$ & 5, 6 & 4, 6, 7 \\
${\rm He^{+} + e^{-}}$ & $\rightarrow$ & ${\rm He^{++} + e^{-} + e^{-}}$ & 1 & 2 \\
${\rm He^{++} + e^{-}}$ & $\rightarrow$ & ${\rm He^{+} + \gamma}$ & 3, 8 & 4, 9 \\
${\rm H + H}$ & $\rightarrow$ & ${\rm H^{+} + e^{-} + H}$ & 10 & 11 \\
${\rm H + He}$ & $\rightarrow$ & ${\rm H^{+} + e^{-} + He}$ & 12 & 11 \\
${\rm H + \gamma}$ & $\rightarrow$ & ${\rm H^{+} + e^{-}}$ & 13 & --- \\
${\rm He + \gamma}$ & $\rightarrow$ & ${\rm He^{+} + e^{-}}$ & 13 & --- \\
${\rm He^{+} + \gamma}$ & $\rightarrow$ & ${\rm He^{++} + e^{-}}$ & 13 & --- \\
\hline
\end{tabular}
\\ Key: 1 -- \citet{1987ephh.book.....J}; 2 -- \citet{1997NewA....2..181A}; 3 -- \citet{1992ApJ...387...95F}; 4 -- \citet{1997MNRAS.292...27H}; 5 -- \citet{1960MNRAS.121..471B}; 6 -- \citet{1973A&A....25..137A}; 7 -- \citet{1981MNRAS.197..553B}; 8 -- \citet{1978ppim.book.....S}; 9 -- \citet{1992ApJS...78..341C}; 10 -- \citet{1987PhRvA..36.3100G}; 11 -- \citet{1991ApJS...76..759L}; 12 -- \citet{1981JChPh..74..314V}; 13 -- see Section~\ref{section:radback}
\end{table}


\subsubsection{Solving and Updating the Network}



\subsection{Cooling} \label{Cooling}


\subsubsection{Primordial Cooling}

\paragraph{Non-equilibrium}

\paragraph{Tabulated}


\subsubsection{Metal Cooling}
- making the tabulated data

- including in the network


\subsection{Heating}


\subsubsection{Photo-ionization Heating}


\subsubsection{Photo-electric Heating}


\subsubsection{Chemical Heating and Cooling}


\subsubsection{Dust Heating}



\subsection{Radiation Backgrounds}
\label{section:radback}
- available models

- including in the solver


\subsection{Applicability and Limitations}
- Density ranges, temperature ranges, optically thin condition

- Remind reader that cooling lengths are potentially very short and
having a fancy model is perhaps not very relevant at 1kpc
resolution. Can give some guidance of how to apply this.


\section{Implementation} \label{methods:code}

In this section, we discuss details of the code, itself, including the
available application program interfaces (APIs), code layout, and how
the existing models and networks may be extended.  For more detailed
information, users should consult online documentation available at
\texttt{https://grackle.readthedocs.org}.

\subsection{Simulation Code API}

The Grackle library provides five main functions to the user for use
in simulation codes: solving the chemistry and cooling (i.e.,
updating the chemical species and internal energy) and calculating the
cooling time, temperature, pressure, and ratio of specific heats.
Before these functions can be called, the code must be initialized
with various user-specified settings.  This initialization process is
also responsible for loading data from external files and calculating
the chemistry and cooling rate tables used by the solvers.  All
Grackle run-time parameters are stored within a C \texttt{struct} of type,
\texttt{chemistry\_data}.  The user initializes this structure by
calling the function, \texttt{set\_default\_chemistry\_parameters}
and supplying a pointer to a \texttt{chemistry\_data}.  The
\texttt{chemistry\_data} pointer is then attached to a globally
viewable pointer called \texttt{grackle\_data}, allowing all run-time
parameters to be accessible without having to store the struct
manually.  Once all parameters are properly set, the user must call
\texttt{initialize\_chemistry\_data} to finalize the initialization
process.  An example of this procedure is shown below:

\vspace{0.5cm}
\begin{minipage}[b]{0.5\linewidth}
\begin{verbatim}

int rval;
chemistry_data my_pars;
rval =
  set_default_chemistry_parameters(&my_pars);

grackle_data.use_grackle = 1;
grackle_data.with_radiative_cooling = 1;
grackle_data.primordial_chemistry = 3;
grackle_data.metal_cooling = 1;
grackle_data.UVbackground = 1;
grackle_data.grackle_data_file = 
  "CloudyData_UVB=HM2012.h5";

rval = initialize_chemistry_data(&my_units);

\end{verbatim}
\end{minipage}

In the above example, the variable \texttt{my\_units} is a C
\texttt{struct} that holds unit conversions from internal code units
to the CGS unit system for quantities such as density, length, and
time.  These are required in order to set up the internal unit system
for the chemistry and cooling rate tables.

Once the Grackle has been initialized, the functionality described
above can be called by the simulation code.  The available functions
are: 1) \texttt{solve\_chemistry} to integrate chemistry and cooling
equations over a specified time step, 2)
\texttt{calculate\_cooling\_time} to calculate the cooling time
($e/(de/dt)$) for each computational element, 3)
\texttt{calculate\_temperature} to calculate the temperature from the
internal energy and chemical species densities, 4)
\texttt{calculate\_pressure} to calculate the gas pressure, and 5)
\texttt{calculate\_gamma} to calculate the ratio of specific heats.
In the Grackle, the ratio of specific heats is only altered from that
of an ideal gas by the presence of H$_{2}$.

For efficiency, Grackle's functions are designed to operate on
multiple computational elements simultaneously.  The user provides
arrays of the required fields to Grackle and their values will be
updated by the chemistry and cooling solvers.  Because the number of
required fields depends on the specific solver being used, Grackle
makes use of another C \texttt{struct} as a means of passing field
arrays to the Grackle functions.  The \texttt{grackle\_field\_data}
\texttt{struct} contains pointers to which can be attached the arrays
of density, internal energy, and any optional fields, such as
individual species densities or the arrays of constant heating rates
(Section \ref{section:constant-heating}).  Since arrays of the
optional fields are only accessed based on run-time parameter
settings, the user has the option of only providing the fields they
wish to use.  The field arrays can be one, two, or three-dimensional,
allowing both Lagrangian particle-based codes and Eulerian mesh-based
codes to provide fields in their native layout.  The
\texttt{grackle\_field\_data} \texttt{struct} contains entries to
specify the field dimensionality as well as to flag certain array
elements as boundary cells which are to be ignored.  An example of
calling the main chemistry solver function on a 16$^{3}$ grid with 3
boundary zones on each side is shown below:

\vspace{0.5cm}
\begin{minipage}[b]{0.5\linewidth}
\begin{verbatim}

grackle_field_data my_fields;

my_fields.grid_rank = 3;
my_fields.grid_dimension = new int[3];
my_fields.grid_start = new int[3];
my_fields.grid_end = new int[3];
for (int i = 0;i < 3;i++) {
  my_fields.grid_dimension[i] = 10;
  my_fields.grid_start[i] = 3;
  my_fields.grid_end[i] = 12;
}

my_fields.density = density_array;
my_fields.internal_energy = energy_array;
my_fields.HI_density = HI_array;
...

// 1 Myr in internal units
double dt = 3.15e7 * 1e6 /
  my_units.time_units;

rval =
  solve_chemistry(&my_units, &my_fields, dt);

\end{verbatim}
\end{minipage}

An added benefit of this approach is that adding new
features which use additional fields will not require a change in the
function signature.  This will, in theory, allow Grackle to maintain
backward compatibility indefinitely.  We note that versions of Grackle
prior to 3.0 did not make use of the \texttt{grackle\_field\_data}
\texttt{struct} and instead required all fields to be provided as
individual arguments.  We acknowledge that the release of
\texttt{Grackle} 3.0 constitutes a significant change to the API, but
one that will ultimately provide more stability moving forward.

\subsection{Pygrackle - Grackle's Python API}

As described above, Grackle's native API is provided through C and Fortran
bindings.  This is particularly useful for simulation codes, as they are
typically written in either C or Fortran compatible languages, but it provides
a barrier to entry for experimental work and testing of Grackle functionality.
Python is a high-level, interpreted language, increasingly used in scientific
computing both as ``glue" code as well as a mechanism for authoring production
scientific codes.  For instance, in 2016, one of the Gordon Bell Prize nominees
utilizes the Python code PyFR to demonstrate extreme scaling of finite element
calculations.

To facilitate access to Grackle functionality, we provide Python bindings to
it, designed to make interacting with chemical rates and rate equations
available to researchers at all stages.  This enables rapid iteration over
different rate coefficients, different initial state vectors, and over
arbitrary time periods.  The bindings are written in Cython
\citep{behnel2010cython} with minimal overhead from Python operations.  Below,
we describe two particular aspects of pygrackle.

\subsubsection{Fluid Container} \label{sec:pyfluid}

Pygrackle provides an object called a ``fluid container.'  When data is passed
to grackle during the course of a simulation, it may be sent as a single zone,
as multiple zones that are organized in a 3D array, or as a 1D ``pencil beam" of
data.  The fluid container object is designed to mimic this, enabling
individuals to ``create" a collection of fluids either by reading them from
disk or constructing them in-memory from NumPy arrays \citep{numpy}.  This
enables calculations of cooling time, chemical evolution, etc, from
analytically-defined gas parcels and distributions.  While this will not take
into account hydrodynamics (unless a package spiritually similar to Grackle,
but for hydrodynamics, is released) it can provide useful and scientifically
relevant information.

The fluid container provides several high-level functions, such as computing
the cooling time, the pressure, and so forth, most of which are usually used
only internally in grackle.  Additionally, this object provides compatibility
with \texttt{yt} \citep{2011ApJS..192....9T}, enabling individuals to load data
with \texttt{yt} and then evolve it through Grackle.  One possible use case for
this is to load in a dataset and use Pygrackle to compute different cooling
times for a collection of gas identified in \texttt{yt} based on different
metallicity assumptions, different chemical rate coefficients, and different
radiation backgrounds.

\subsubsection{Evolution Models} \label{sec:pyevolve}

In addition to providing access to \texttt{Grackle}'s primary
functionality, Pygrackle also features a set of convenience functions
to evolve a fluid container forward in time following simple models.
These functions return a dictionary of arrays of all fluid quantities
(i.e., species densities, internal energy, temperature, etc.) for time
values of the evolution.  These functions can be used in semi-analytic
models that require knowledge of the thermal evolution of gas under
different conditions.  Examples of use of are discussed in
\ref{sec:testing}.

The simplest of these evolves a fluid container assuming a constant
density model.  The \texttt{evolve\_constant\_density} function takes
an initialized fluid container and repeatedly calls \texttt{Grackle}'s
\texttt{solve\_chemistry} function until a specified stopping time or
temperature has been reached.  For each iteration, the timestep is
taken to be a fraction of the local cooling time, specifiable by the
user and defaulting to 0.01.

The second of these functions models the evolution of a parcel of gas
collapsing due to self-gravity.  The \texttt{evolve\_freefall} function
closely follows the one-zone free-fall collapse model introduced by
\citet{2000ApJ...534..809O} and modified by
\citet{2005ApJ...626..627O} to include the effects of thermal pressure
support.  The gas density evolves following the modified collapse
model of \citet{2005ApJ...626..627O}, given by
\begin{equation}
\frac{d \rho}{dt} = \frac{\rho}{t_{col}},
\end{equation}
where $t_{\rm col}$ is the collapse time-scale expressed as
\begin{equation}
t_{col} = \frac{t_{ff}}{\sqrt{1-f}},
\end{equation}
and the free-fall time is given by
\begin{equation}
t_{ff} = \sqrt{\frac{3 \pi}{32 G \rho}}.
\end{equation}
Thermal pressure forces, which act to slow the collapse of the cloud,
are modeled by the factor $f$, which is expressed as
\begin{equation}
f = \left\{
\begin{array}{lr}
0, & \gamma < 0.83,\\
0.6 + 2.5(\gamma - 1) - 6.0(\gamma - 1)^2, & 0.83 < \gamma < 1,\\
1.0 + 0.2(\gamma - 4/3) - 2.9(\gamma - 4/3)^2, & \gamma > 1,
\end{array}
 \right.
\end{equation}
where the effective adiabatic index, $\gamma$, is
\begin{equation}
\gamma \equiv \frac{\partial \ln p}{\partial \ln \rho}.
\end{equation}
As the density increases, the internal energy is altered by a
combination of adiabatic compression and radiative cooling, computed
by Grackle, and is given by
\begin{equation}
\frac{de}{dt}= -p \frac{d}{dt} \frac{1}{\rho} - {\Lambda},
\label{eq:energy}
\end{equation}
where the pressure is given by
\begin{equation}
p = \frac{\rho k T}{\mu m_{\rm H}},
\end{equation}
the specific internal energy is
\begin{equation}
e = \frac{1}{\gamma_{\rm ad} -1} \frac{k T}{\mu m_{\rm H}},
\label{eq:defen}
\end{equation}
$\gamma_{\rm ad}$ is the true adiabatic index, and $\Lambda$ is the
radiative cooling rate.  For each iteration of this function, the
timestep is taken to be a fraction of the local free-fall time,
specifiable by the user and defaulting to 0.01.  An example use of
this function is shown in Section \ref{sec:free-fall-test}.

\subsection{Code Structure} \label{Code_Structure}

\subsection{Adding New Models/Rates}

Adding new rates to Grackle involves modifying the values stored in the rate coefficients and 
additionally defining a new network for both the chemistry and if desired also the cooling. As it stands the
code is not sufficiently modular to allow this task to be completed with significant ease and some
hands on manipulation of the code structure is required (see \S \ref{Future_Directions} for 
future improvements in this regard). \\
\indent When considering a manipulation of the Grackle code base an important consideration is that the structure
of the current code base is set up to realize either an equilibrium chemistry model using cooling tables
derived from Cloudy or a non-equilibrium cooling model based on a six, nine or twelve species cooling model. 
If your wish is to implement a different equilibrium cooling model then the process is relatively straight
forward. Grackle reads the cloudy cooling and interpolates the data along one, two or three dimensions as
appropriate. By substituting in a different cooling table the behavior of the cooling for a given species
can be manipulated. The format of the cooling table should however match the format of the Cloudy cooling
tables. \\
\indent If on the other-hand your intention is to manipulate the non-equilibrium cooling behavior then more
code manipulation will be required. Grackle assumes that non-equilibrium cooling models are hierarchical
with the nine species model containing the six species model plus three additional components for example. 
If the intention is to expand the network to include a fourth model containing the twelve species model as 
a subset then the procedure is straight forward. However, if a different chemical network is envisaged 
containing some species already in the twelve species model and some not then this will require more care. 
The flow of the chemistry network is controlled by the primordial\_chemistry flag, 
primordial\_chemistry = 1 for the six species model, primordial\_chemistry = 2 for the nine species model and
primordial\_chemistry = 3 for the twelve species model. In general through out the code an if clause of 
if(primordial\_chemistry > 0) is followed by else if (primordial\_chemistry > 1) which means that instructions
are carried out consecutively assuming one model is a subset of a higher order model.
For example say we wish to implements a 10 species model using the current 9 species set and one other element. 
Two options exist in this case. We could implement a 13 species model and set primordial\_species = 4 and 
update the code accordingly to deal with an additional species setting the Deuterium species to zero in input.
This would be straight forward but somewhat wasteful. The alternative option would be to adjust the 
primordial\_species = 2 criteria to include an extra species. This species would also need to be added into
Grackle API. In both of these cases the hierarchical nature of the code would be preserved. \\
%\indent In the following 
%example we will discuss the more straight forward case of a 15 species model. The new model will 
%contain the original twelve species model plus new three chemical elements to evolve - 
%primordial\_chemistry = 4 as a result becomes our fictional 15 species model.

\subsubsection{Updating the Rate Coefficients}
As described in \S \ref{Code_Structure} the rates are allocated and declared in initialize\_chemistry.c. If the
intention is to use your own rate coefficients (say based on more recent experimental data) then 
it would still be prudent to make use of the already existing rate coefficient arrays (k array) as declared
in initialize\_chemistry.c. The rates are populated in calc\_rates\_g.F. Within this function the rates 
are fully described. For example k1 is the rate coefficient for the collisional ionization of neutral 
Hydrogen (k1: HI + e -> HII + 2e). If a significant number of new rate coefficients are desired then 
the most expedient approach would be to insert a \#define preprocessor directive into calc\_rates\_g.F
into which your appropriate function call can be inserted. In this case upon compilation your new function 
will populate all of the rate coefficients required for your new network. Your function can use the 
routine coll\_rates\_g as a template for how to implement a new set of rate coefficients. Any rates that are not
set in the new routine will be set to a very small number and will therefore not effect the calculation.\\
\indent For any additional rate coefficients that are required the k array needs to be declared and allocated
in initialize\_chemistry.c. These rates can then be used to model the chemical behavior of the new species
as a function of temperature and density as required. Furthermore, in the function lookup\_cool\_rates1d\_g 
in solve\_rate\_cool\_g.F the interpolation of the new rates will need to be added. 


\subsection{Updating the chemistry network}
If in addition (or independently) you wish to update the chemical network to either include or exclude
reactions then a new rate network will be required. As a template the function step\_rate in 
solve\_rate\_cool\_g.F can be used. This function currently evolves both the six, nine and twelve species 
models. To modify this network you can either create a new chemistry solver following the 
existing solver's methods as a guide or create a new routine which solves the chemical network. If the 
new network is simply an addition to the existing network (e.g. a 15 species model) then the easiest 
option is to simply augment this network with the three extra species using the appropriate 
interactions. The species will then be evolved until they converge.

\subsection{Updating the cooling network}
Apart from the chemical network the cooling network may also be modified. As discussed previously if the intention
is to implement a new cooling table then the changes are straight forward. Once the tables are formatted correctly
Grackle will the new values to calculate a new temperature. If instead we want to modify the cooling rate due 
to emission line cooling from given species this is handled in the cool1d\_multi\_g.F. As discussed in \S
\ref{Cooling} line emission cooling is determined using collisional excitation, collisional ionization and 
recombination rates. If the intention is to update/modify existing rates then the cooling rates are also set
in calc\_rates\_g.F and can be modified here. If new cooling rates are required from another species 
whose cooling properties are important for the network then the rates can be added here also. Any new arrays
required in this case will also need to be declared and initialized in initialize\_chemistry.c similar to the
rate coefficients. The rates can then be interpolated to the required temperature values in calc\_rates\_g.F 
following the examples there. Once the new cooling rate has been determined it's values needs to be added
to the edot array so that the cooling effects of this new species is properly account for.




\section{Profiling and Testing}
\label{sec:profiling-and-testing}



\subsection{Testing Framework}

Testing a library like Grackle, with its mix of \texttt{Fortran}, and
\texttt{C++} code, can be difficult. In the interest of ultimately improving
test coverage and make it easier to prototype tests, Grackle employs a
python-based testing framework that makes heavy use of the \texttt{pygrackle}
python wrapper for the Grackle. The tests are orchestrated using the
\texttt{py.test}\footnote{http://pytest.org/} package.

Currently there are two different types of tests in the Grackle test suite: unit
tests and answer tests. A unit test in Grackle compares the results of a
calculation using Grackle to some set of ``correct'' answers that are known
\textit{a priori}. The unit tests currently implemented in grackle test that the
unit system is behaving correctly (see Section~\ref{sec:unit-test}) and that the
ionization equilibrium for a primordial gas agrees with the analytic prediction
using the rates implemented in Grackle (see Section~\ref{sec:primordial-test}).

The answer tests consist of a set of example calculations where each calculation
writes out a summary plot as well as an \texttt{hdf5} dataset that is loadable
by the \texttt{yt} package. Known ``correct'' answers for the summary plot and
\texttt{yt}-loadable dataset are saved in the repository so that code changes in
Grackle can be tested to ensure that andwers produced by the library do not
change. This process does not prevent \textit{incorrect} answers from being
generated initially, but it does prevent the answers from changing as the code
is developed. Incorrect answers are prevented by manually inspecting test
answers when the test is first added to the codebase. If subsequently a bug is
discovered and the test output changes, then the test answer must also be
manually updated. Currently Grackle contains answer tests calculations for the
cooling rate (see Section~\ref{sec:cooling-rate-test}) in a uniform density
cloud at a fixed time, a calculation of the density and temperature of a gas
cloud undergoing free-fall collapse (see Section~\ref{sec:free-fall-test}), and
a calculation for the temperature evolution of a uniform-density cloud (see
Section~\ref{sec:uniform-cooling-test}).  The answers test are run several times
using different input physics to ensure Grackle's solvers are well-exercised by
the tests.

\subsubsection{unit test: units}
\label{sec:unit-test}

\subsubsection{unit test: ionization equailibrium of a primordial gas}
\label{sec:primordial-test}

\subsubsection{Answer test: cooling rate test}
\label{sec:cooling-rate-test}

\subsubsection{Answer test: free-fall cooling test}
\label{sec:free-fall-test}

\subsubsection{Answer test: uniform cooling test}
\label{sec:uniform-cooling-test}

\subsubsection{Other tests?}


\subsection{Performance}


\subsubsection{Optimization Strategy}


\subsubsection{Speed Test}


\section{Summary} \label{sec:summary}

\subsection{Applicability and Limitations}

- Density ranges, temperature ranges, optically thin condition

- Remind reader that cooling lengths are potentially very short and
having a fancy model is perhaps not very relevant at 1kpc
resolution. Can give some guidance of how to apply this.

\begin{figure}
  \centering
  \includegraphics[width=0.48\textwidth]{valid_range.pdf}
  \caption{
    The appropriate density range for different versions of the
    Grackle solver.  The high density metal cooling table (bottom)
    explicitly spans the density range, 10$^{-6}$ cm $^{-3}$ $< n <$
    10$^{12}$ cm $^{-3}$, but extrapolation down to $n = 10^{-10}$ cm
    $^{-3}$ is still valid, as indicated by the dashed line.
  } \label{fig:valid-range}
\end{figure}

\begin{figure*}
  \centering
  \includegraphics[width=0.84\textwidth]{cooling_length.pdf}
  \caption{
    The cooling length, defined as the product of the Jeans length and
    the cooling time, as a function of number density and temperature
    for a gas with solar metallicity exposed to a radiation field
    defined by the model of \citet{2012ApJ...746..125H} at z = 0.  The
    narrow line extending from the middle, left to the bottom, right
    represents the temperature where heating and cooling are
    balanced.  Above this line, the gas is being cooled while below
    the line it is being heated.
  } \label{fig:cooling-length}
\end{figure*}

\subsection{codes which have Grackle implementations}


\subsection{Future Directions} \label{Future_Directions}

\subsubsection{Including new rates and models in Grackle}
\jr{} The current code structure is highly integrated. This makes introducing new rates for the 
chemical network or cooling network a rather intricate task requiring multiple changes throughout the code. 
Apart from the fact that this is more time consuming it is also much more error prone. In a future release of the 
code the modularity of the code will be increased greatly. There will be a function to populate the species 
rate coefficients and a function to populate the cooling coefficients. Seperate template files can then be 
updated by a developer wishing to use their own rates. This file can then be included in the build and a flag
set to indicate the new rates be used in place of the old rates. Furthermore, a similar method will be 
implemented for solving the network. A template network solver will be available which the developer can use to 
implement a new network with a developer determined number of species. The developer will be responsible for
updating only three files to achieve a solution to their own chemical network. 

\subsubsection{Connection to Radiative Transfer}

\subsubsection{Multiple element cooling}

\subsubsection{Other heating sources?}


\section*{Acknowledgements}

BDS would like to thank Michael Kuhlen for his work on the code in the
early stages of the project; the organizers of the AGORA project, Ji-hoon
Kim, Joel Primack, and Piero Madau for the original motivation for the
\texttt{Grackle} project; as well as Nick Gnedin, James Wadsley, and
Romeel Dav$\acute{\rm e}$ for providing useful suggestions for
additional functionality.  GLB acknowledges support from
NASA grant NNX15AB20G and NSF grant 1312888. SCOG acknowledges support
from the Deutsche Forschungsgemeinschaft via SFB 881, ``The Milky Way System'' 
(sub-projects B1, B2 and B8) and SPP 1573, ``Physics of the Interstellar Medium'' 
(grant number GL 668/2-1), and from the European Research Council under the 
European Community's Seventh Framework Programme (FP7/2007-2013) via the ERC 
Advanced Grant STARLIGHT (project number 339177). AE is supported by a NSF Graduate
Research Fellowship grant No. DGE-16-44869. We also thank the NSF for computational
resources provided through the XSEDE program. JAR acknowledges support through the STFC 
capital grant ST/L00075X/1.



%%%%%%%%%%%%%%%%%%%%%%%%%%%%%%%%%%%%%%%%%%%%%%%%%%

%%%%%%%%%%%%%%%%%%%% REFERENCES %%%%%%%%%%%%%%%%%%

\footnotesize{
  \bibliographystyle{mnras}
  \citestyle{mnras}
  \bibliography{ms}
}

\bsp
\label{lastpage}
\end{document}
