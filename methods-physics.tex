\section{Chemistry and Cooling Methods} \label{sec:physics}



\subsection{Primordial Chemistry}
The treatment of primordial chemistry (i.e.\ the chemistry of metal-free gas) used in Grackle is closely based on the
treatment in the Enzo AMR code \citep{2014ApJS..211...19B}, although Grackle accounts for a few processes that are not included 
in the latest version of Enzo available at the time of writing (version ***). The Enzo primordial chemistry itself is based 
originally on the work of \citet{1997NewA....2..181A} and \citet{1997NewA....2..209A}, although the current version has been
modified substantially compared to the original Abel~et~al.\ network. In this section, we describe in detail the chemistry 
included in Grackle and discuss how the resulting set of chemical rate equations is solved. 

\subsubsection{Chemistry Network}
Grackle provides three different primordial chemistry networks, differing in the number of chemical species that they include. 
The six-species network is designed for modelling atomic and/or ionized gas, and tracks the abundances of the species
H, H$^{+}$, He, He$^{+}$, He$^{++}$ and e$^{-}$. The reactions included in this network are listed in Table ***, along with
the sources used for the various reaction rate coefficients. 



\subsubsection{Solving and Updating the Network}



\subsection{Cooling}


\subsubsection{Primordial Cooling}

\paragraph{Non-equilibrium}

\paragraph{Tabulated}


\subsubsection{Metal Cooling}
- making the tabulated data

- including in the network


\subsection{Heating}


\subsubsection{Photo-ionization Heating}


\subsubsection{Photo-electric Heating}


\subsubsection{Chemical Heating and Cooling}


\subsubsection{Dust Heating}



\subsection{Radiation Backgrounds}
- available models

- including in the solver


\subsection{Applicability and Limitations}
- Density ranges, temperature ranges, optically thin condition

- Remind reader that cooling lengths are potentially very short and
having a fancy model is perhaps not very relevant at 1kpc
resolution. Can give some guidance of how to apply this.
