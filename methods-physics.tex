\section{Chemistry and Cooling Methods} \label{sec:physics}



\subsection{Primordial Chemistry}
Sample citation is \citet{1997NewA....2..181A}.

\subsubsection{Chemistry Network}


\subsubsection{Solving and Updating the Network}



\subsection{Cooling} \label{Cooling}


\subsubsection{Primordial Cooling}

\paragraph{Non-equilibrium}

\paragraph{Tabulated}
In the simplest mode, the cooling and heating due to the primordial
elements can be calculated using tables of pre-computed values under
the assumption of ionization equilibrium.  When the gas is assumed to
have no incident radiation, this is the case of collisional ionization
equilibrium (CIE), and the cooling rate is solely a function of
temperature.  If radiation is present, the cooling rate under
ionization equilibrium for a fixed spectral shape and intensity is a
function of density and temperature.  


The
nature of the tables and the process by which they are created is
discussed in Section \ref{sec:cooling-tables}.  

\subsubsection{Metal Cooling}

\paragraph{Constructing Cooling Tables} \label{sec:cooling-tables}


\subsection{Heating}


\subsubsection{Photo-ionization Heating}


\subsubsection{Photo-electric Heating}


\subsubsection{Chemical Heating and Cooling}


\subsubsection{Dust Heating}



\subsection{Radiation Backgrounds}
- available models

- including in the solver


\subsection{Applicability and Limitations}
- Density ranges, temperature ranges, optically thin condition

- Remind reader that cooling lengths are potentially very short and
having a fancy model is perhaps not very relevant at 1kpc
resolution. Can give some guidance of how to apply this.
