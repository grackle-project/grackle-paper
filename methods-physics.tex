\section{Chemistry and Cooling Methods} \label{sec:physics}



\subsection{Primordial Chemistry}
The treatment of primordial chemistry (i.e.\ the chemistry of metal-free gas) used in Grackle is closely based on the
treatment in the Enzo AMR code \citep{2014ApJS..211...19B}, although Grackle accounts for a few processes that are not included 
in the latest version of Enzo available at the time of writing (version 2.4). The Enzo primordial chemistry itself is based 
originally on the work of \citet{1997NewA....2..181A} and \citet{1997NewA....2..209A}, although the current version has been
modified substantially compared to the original Abel~et~al.\ network. In this section, we describe in detail the chemistry 
included in Grackle and discuss how the resulting set of chemical rate equations is solved. 

\subsubsection{Chemistry Network}
Grackle provides three different primordial chemistry networks, differing in the number of chemical species that they include. 
The choice of chemical network is controlled by the primordial\_chemistry parameter. Setting primordial\_chemistry = 1 selects the
six species network, which tracks the abundances of the species H, H$^{+}$, He, He$^{+}$, He$^{++}$ and e$^{-}$ and is 
designed for modelling atomic and/or ionized gas. Setting primordial\_chemistry = 2 selects the nine species network. This
includes all of the species and reactions included in the six species network, but adds molecular hydrogen (H$_2$), plus the
two ions primarily responsible for its formation in primordial gas (H$^{-}$ and H$_{2}^{+}$). Finally, setting primordial\_chemistry = 3
selects the twelve species network, which is an extension of the nine species model that includes D, D$^{+}$ and HD.

\begin{table}
\caption{Chemical reactions in the six species network \label{tab:six}}
\begin{tabular}{lclcc}
\hline
\multicolumn{3}{c}{Reaction} & \multicolumn{2}{c}{Reference} \\
& & & Data & Fit \\
\hline
${\rm H + e^{-}}$ & $\rightarrow$ & $\rm{H^{+} + e^{-} + e^{-}}$ & 1 & 2 \\
${\rm H^{+} + e^{-}}$ & $\rightarrow$ & $\rm{H + \gamma} $ & 3 & 2, 4\\
${\rm He + e^{-}}$ & $\rightarrow$ & ${\rm He^{+} + e^{-} + e^{-}}$ & 1 & 2 \\
${\rm He^{+} + e^{-}}$ & $\rightarrow$ & ${\rm He + \gamma}$ & 5, 6 & 4, 6, 7 \\
${\rm He^{+} + e^{-}}$ & $\rightarrow$ & ${\rm He^{++} + e^{-} + e^{-}}$ & 1 & 2 \\
${\rm He^{++} + e^{-}}$ & $\rightarrow$ & ${\rm He^{+} + \gamma}$ & 3, 8 & 4, 9 \\
${\rm H + H}$ & $\rightarrow$ & ${\rm H^{+} + e^{-} + H}$ & 10 & 11 \\
${\rm H + He}$ & $\rightarrow$ & ${\rm H^{+} + e^{-} + He}$ & 12 & 11 \\
${\rm H + \gamma}$ & $\rightarrow$ & ${\rm H^{+} + e^{-}}$ & 13 & --- \\
${\rm He + \gamma}$ & $\rightarrow$ & ${\rm He^{+} + e^{-}}$ & 13 & --- \\
${\rm He^{+} + \gamma}$ & $\rightarrow$ & ${\rm He^{++} + e^{-}}$ & 13 & --- \\
\hline
\end{tabular}
\\ Key: 1 -- \citet{1987ephh.book.....J}; 2 -- \citet{1997NewA....2..181A}; 3 -- \citet{1992ApJ...387...95F}; 4 -- \citet{1997MNRAS.292...27H}; 5 -- \citet{1960MNRAS.121..471B}; 6 -- \citet{1973A&A....25..137A}; 7 -- \citet{1981MNRAS.197..553B}; 8 -- \citet{1978ppim.book.....S}; 9 -- \citet{1992ApJS...78..341C}; 10 -- \citet{1987PhRvA..36.3100G}; 11 -- \citet{1991ApJS...76..759L}; 12 -- \citet{1981JChPh..74..314V}; 13 -- see Section~\ref{section:radback}
\end{table}

The reactions included in the six species network are listed in Table~\ref{tab:six}. The rate coefficients for these reactions are implemented in
Grackle using simple temperature-dependent analytical fits. In the Table, we list the references from which we take these fits, and also the 
references that are the original sources of the theoretical or experimental  data on which these fits are based.

A few of the reactions listed in Table~\ref{tab:six} deserve further comment:
\begin{enumerate}
\item[(i)] By default, the recombination of
H$^{+}$, He$^{+}$ and He$^{++}$ is modelled using the case A recombination rate coefficients. However, the case B
rate coefficients can instead be selected by setting casebrates = 1. The additional complication that photons from the recombination of
He$^{+}$ can bring about the photoionization of hydrogen \citep[discussed in some detail in][]{1989agna.book.....O} is not accounted
for, but in most circumstances this will only lead to a small error in the H$^{+}$ and He$^{+}$ abundances. 
\item[(ii)] The rates of the photoionization reactions are not calculated internally by Grackle, but instead are specified in an input data file,
as described in more detail in Section~\ref{section:radback}.
\item[(iii)] The six species network implemented in Grackle includes two additional reactions that were not part of the original \citet{1997NewA....2..181A}
six species network: the collisional ionization of H by collisions with H and He atoms. Often, these reactions are unimportant. However, they 
can become competitive with the collisional ionization of H by electrons if the fractional ionization of the gas is very low \citep[see e.g.][for an example of when this can be important]{2015MNRAS.451.2082G}.
\end{enumerate}

\begin{table}
\caption{Chemical reactions in the nine species network \label{tab:nine}}
\begin{tabular}{lclcc}
\hline
\multicolumn{3}{c}{Reaction} & \multicolumn{2}{c}{Reference} \\
& & & Data & Fit \\
\hline
${\rm H + e^{-}}$ & $\rightarrow$ & ${\rm H^{-} + \gamma}$ & 1 & 2 \\
${\rm H^{-} + H}$ & $\rightarrow$ & ${\rm H_{2} + e^{-}}$ & 3 & 3 \\
${\rm H + H^{+}}$ & $\rightarrow$ & ${\rm H_{2}^{+} + \gamma}$ & 4 & 5 \\
${\rm H_{2}^{+} + H}$ & $\rightarrow$ & ${\rm H_{2} + H^{+}}$ & 6 & 6 \\
${\rm H_{2} + H^{+}}$ & $\rightarrow$ & ${\rm H_{2}^{+} + H}$ & 7 & 8 \\
${\rm H_{2} + e^{-}}$ & $\rightarrow$ & ${\rm H + H + e^{-}}$ & 9 & 9 \\
${\rm H_{2} + H}$ & $\rightarrow$ & ${\rm H + H + H}$ & 10 & 10 \\
${\rm H^{-} + e^{-}}$ & $\rightarrow$ & ${\rm H + e^{-} + e^{-}}$ & 11 & 12 \\
${\rm H^{-} + H}$ & $\rightarrow$ & ${\rm H + e^{-} + H}$ & 11 & 12 \\
${\rm H^{-} + H^{+}}$ & $\rightarrow$ & ${\rm H + H}$ & 13 & 14 \\
${\rm H^{-} + H^{+}}$ & $\rightarrow$ & ${\rm H_{2}^{+} + e^{-}}$ & 15 & 12, 16 \\
${\rm H_{2}^{+} + e^{-}}$ & $\rightarrow$ & ${\rm H + H}$ & 17 & 12 \\
${\rm H_{2}^{+} + H^{-}}$ & $\rightarrow$ & ${\rm H_{2} + H}$ & 18 & 18 \\ 
${\rm H + H + H}$ & $\rightarrow$ & ${\rm H_{2} + H}$ & 19  & 19  \\
${\rm H + H + H_{2}}$ & $\rightarrow$ & ${\rm H_{2} + H_{2}}$ & 20, 21 & 22 \\
${\rm H^{-} + \gamma}$ & $\rightarrow$ & ${\rm H + e^{-}}$ & 23 & --- \\
${\rm H_{2}^{+} + \gamma}$ & $\rightarrow$ & ${\rm H + H^{+}}$ & 23 & --- \\
${\rm H_{2} + \gamma}$ & $\rightarrow$ & ${\rm H_{2}^{+} + e^{-}}$ & 23 & --- \\
${\rm H_{2}^{+} + \gamma}$ & $\rightarrow$ & ${\rm H^{+} + H^{+} + e^{-}}$ & 23 & --- \\
${\rm H_{2} + \gamma}$ & $\rightarrow$ & ${\rm H + H}$ & 23 & --- \\
\hline
\end{tabular}
\\ Note: the nine species network also includes all of the reactions listed in Table~\ref{tab:six}.
\\ Key: 1 -- \citet{1979MNRAS.187P..59W}; 2 -- \citet{1998ApJ...509....1S}; 3 -- \citet{2010Sci...329...69K}; 4 -- \citet{1976PhRvA..13...58R}; 5 -- \citet{2015MNRAS.446.3163L}; 6 -- \citet{1979JChPh..70.2877K}; 7 -- \citet{2002PhRvA..66d2717K}; 8 --- \citet{2004ApJ...606L.167S,2004ApJ...607L.147S}; 9 -- \citet{2002PPCF...44.1263T}; 10 -- \citet{1996ApJ...461..265M}; 11 -- \citet{1987ephh.book.....J}; 12 -- \citet{1997NewA....2..181A}; 13 -- \citet{1986JPhB...19L..31F}; 14 -- \citet{1999MNRAS.304..327C}; 15 -- \citet{1978JPhB...11L.671P}; 16 -- \citet{1987ApJ...318...32S}; 17 -- \citet{1994ApJ...424..983S}; 18 -- \citet{1987IAUS..120..109D}; 19 -- See text; 20 -- \citet{1962JChPh..36.2923S}; 21 -- \citet{1970JChPh..53.4395H}; 22 -- \citet{1983JPCRD..12..531C}; 23 -- see Section~\ref{section:radback}.
\end{table}

The nine species network includes all of the reactions in Table~\ref{tab:six} plus the additional reactions listed in Table~\ref{tab:nine}. Again,
a couple of reactions deserve further discussion:
\begin{enumerate}
\item[(i)] The treatment of H$_{2}$ collisional dissociation by H atom collisions now used in Grackle is taken from \citet{1996ApJ...461..265M} and 
accounts for both the temperature and the density dependence of this process. It therefore remains valid in the high density limit, where the H$_{2}$
level populations approach their local thermodynamical equilibrium (LTE) values. This is important, because H$_{2}$ is far more susceptible to
collisional dissociation in this limit than when it is solely in the vibrational ground state. It should also be noted that the treatment of this process
in Grackle accounts for effects of dissociative tunneling as well as direct collisional dissociation; previously, Enzo only accounted for the latter
process and hence underestimated the dissociation rate at low temperatures \citep{2014MNRAS.443.1979L,2015MNRAS.451.2082G}
\item[(ii)] In view of the considerable uncertainty in the rate of the three-body reaction
\begin{equation}
{\rm H + H + H} \rightarrow {\rm H_{2} + H},
\end{equation}
discussed in detail in \citet{2008AIPC..990...25G} and \citet{2011ApJ...726...55T}, Grackle provides several different rate coefficients for this process. The user can
select which of these rate coefficients to adopt by means of the three\_body\_rate parameter. The options are
\begin{enumerate}
\item[{\bf 0}:]  Rate coefficient from \citet{2002Sci...295...93A}, based on an extrapolation from low temperature calculations by \citet{1987JChPh..87..314O}. This is the default option.
\item[{\bf 1}:]  Rate coefficient from \citet{1983ApJ...271..632P}, derived using detailed balance applied to the H$_{2}$ collisional dissociation rate measured by \citet{1967JChPh..47...54J}.
\item[{\bf 2}:]  Rate coefficient recommended by \citet{1983JPCRD..12..531C}, based on a survey of the experimental data available at that time. 
\item[{\bf 3}:]  Rate coefficient from \citet{2007MNRAS.377..705F}, also derived from \citet{1967JChPh..47...54J} using detailed balance, but with a different treatment of the H$_{2}$ partition
function. 
\item[{\bf 4}:]  Rate coefficient from \citet{2008AIPC..990...25G}, derived from the \citet{1996ApJ...461..265M} high-density H$_2$ collisional dissociation rate using detailed balance
\item[{\bf 5}:]  Rate coefficient computed directly by \citet{2013ApJ...773L..25F}.
\end{enumerate}
\end{enumerate}

Finally, the twelve species network includes the reactions in Tables~\ref{tab:six} and \ref{tab:nine}, plus a small number of additional reactions involving
D$^{+}$, D and HD, listed in Table~\ref{tab:12}. The intent of the twelve species network is to allow the HD abundance of the gas to be tracked accurately,
since in cold gas HD can become a more effective coolant than H$_{2}$ despite its much lower fractional abundance \citep[see e.g.][]{2006MNRAS.366..247J,2008ApJ...685....8M}. It is therefore necessary to include only a small number of reactions, as the direct conversion of H$_{2}$ to HD by collisions with D$^{+}$ and D, together with the
corresponding inverse reactions generally dominate the production and destruction of HD.

Two reactions in the twelve species network require further discussion:
\begin{enumerate}
\item[(i)] The rate coefficient that we adopt for the reaction
\begin{equation}
{\rm HD + H} \rightarrow {\rm H_{2} + D}
\end{equation}
is an analytical fit presented in \citet{2002P&SS...50.1197G}, based on data from \citet{1959JChPh..31.1359S}. However, this fit blows up
at temperatures $T < 100$~K, yielding an unphysically large value for the rate coefficient. We therefore follow \citet{2007MNRAS.376..709R}
and \citet{2008ApJ...685....8M} and assume that the rate at $T < 100$~K is the same as the rate at $T = 100$~K. Note that as this reaction
proceeds extremely slowly at temperatures below as few hundred K, this is unlikely to be a significant source of error. 
\item[(ii)] We assume that the rate coefficient for the associative detachment of H$^{-}$ by D
\begin{equation}
{\rm D + H^{-} \rightarrow  HD + e^{-}}
\end{equation}
is the same as for the corresponding reaction between H and H$^{-}$, since measurements by \citet{2012PhRvA..86c2714M} suggest that
there is not a significant isotope effect for this reaction. However, in the solver, we multiply the rate coefficient by a factor of two when
computing the HD formation rate to account approximately for the contribution from the reaction
\begin{equation}
{\rm H + D^{-} \rightarrow  HD + e^{-}}.
\end{equation}
Note that we do not explicitly include this reaction in our network because it would require us to track the abundance of the D$^{-}$ ion,
thereby adding significant additional complexity to the model for only a marginal increase in accuracy.
\end{enumerate}

\begin{table}
\caption{Additional reactions included in the twelve species network \label{tab:12}}
\begin{tabular}{lclcc}
\hline
\multicolumn{3}{c}{Reaction} & \multicolumn{2}{c}{Reference} \\
& & & Data & Fit \\
\hline
${\rm H^{+} + D}$ & $\rightarrow$ & ${\rm H + D^{+}}$ & 1, 2 & 3 \\
${\rm D^{+} + H}$ & $\rightarrow$ & ${\rm D + H^{+}}$ & 1, 2 & 3 \\
${\rm H_{2} + D^{+}}$ & $\rightarrow$ & ${\rm HD + H^{+}}$ & 4 & 5 \\
${\rm HD + H^{+}}$ & $\rightarrow$ & ${\rm H_{2} + D^{+}}$ & 4 & 5 \\
${\rm H_{2} + D}$ & $\rightarrow$ & ${\rm HD + H}$ & 6 & 7 \\
${\rm HD + H}$ & $\rightarrow$ & ${\rm H_{2} + D}$ & 8 & 5, 9 \\
${\rm D + H^{-}}$ & $\rightarrow$ & ${\rm HD + e^{-}}$ & 10, 11 & 10  \\
\hline
\end{tabular}
\\ Note: the twelve species network also includes all of the reactions listed in Tables~\ref{tab:six} \& \ref{tab:nine}.
\\ Key: 1 -- \citet{1999PhRvL..83.4041I}; 2 -- \citet{2000PhRvA..62d2706Z}; 3 -- \citet{2002ApJ...566..599S};
4 -- \citet{1982sasp.nasa..304G}; 5 -- \citet{2002P&SS...50.1197G}; 6 -- \citet{2003PhRvL..91f3201M}; 7 -- \citet{2011ApJ...727..110C}; 
8 -- \citet{1959JChPh..31.1359S}; 9 -- \citet{2007MNRAS.376..709R}; 10 -- \citet{2010Sci...329...69K}; 11 -- \citet{2012PhRvA..86c2714M}
\end{table}

\subsubsection{Solving and Updating the Network}



\subsection{Cooling} \label{Cooling}


\subsubsection{Primordial Cooling}

\paragraph{Non-equilibrium}

\paragraph{Tabulated}
In the simplest mode, the cooling and heating due to the primordial
elements can be calculated using tables of pre-computed values under
the assumption of ionization equilibrium.  When the gas is assumed to
have no incident radiation, this is the case of collisional ionization
equilibrium (CIE), and the cooling rate is solely a function of
temperature.  If radiation is present, the cooling rate under
ionization equilibrium for a fixed spectral shape and intensity is a
function of density and temperature.  The process by which they are
created is discussed in Section \ref{sec:cooling-tables}.  For the
primordial elements, tables exist for the cooling rate, $\Lambda$;
the heating rate, $\Gamma$; and the mean molecular weight, $\mu$, of
the gas as a function of temperature and, optionally, density.  Since
simulation codes typically solve for the internal energy of the gas
instead of the temperature, it is necessary to convert one to the
other by
\begin{equation} \label{eqn:e-T}
e = \frac{k T}{(\gamma - 1)\ \mu m_{H}},
\end{equation}
where $k$ is the Boltzmann constant, $T$ is the temperature, $e$ is
the specific internal energy, $\gamma$ is the adiabatic index of an
ideal gas, and $m_{H}$ is the Hydrogen mass.  Since $\mu$ is also a
function of temperature, we solve Equation \ref{eqn:e-T} iteratively
with an initial guess of $\mu = 1$.  The temperature calculated using
the initial guess for $\mu$ is then used as an input to the table of
$\mu(T)$ 


\subsubsection{Metal Cooling}

\paragraph{Constructing Cooling Tables} \label{sec:cooling-tables}


\subsection{Heating}


\subsubsection{Photo-ionization Heating}


\subsubsection{Photo-electric Heating}


\subsubsection{Chemical Heating and Cooling}


\subsubsection{Dust Heating}



\subsection{Radiation Backgrounds}
\label{section:radback}
- available models

- including in the solver


\subsection{Applicability and Limitations}
- Density ranges, temperature ranges, optically thin condition

- Remind reader that cooling lengths are potentially very short and
having a fancy model is perhaps not very relevant at 1kpc
resolution. Can give some guidance of how to apply this.
